\documentclass[20150903-160354-rs2.2-MarksMathNotebook.tex]{subfiles}

\begin{document}
%-=-=-=-=-=-=-=-=-=-=-=-=-=-=-=-=-=-=-=-=-=-=-=-=
%
%	CHAPTER
%
%-=-=-=-=-=-=-=-=-=-=-=-=-=-=-=-=-=-=-=-=-=-=-=-=

\chapter{Factoring Univariate Trinomials}

%-=-=-= EXAMPLE
\begin{example}[id:20151015-184212] \label{20151015-184212}\index{Example!20151015-184212} \hfill \\
Simplify by factoring $x^2+7x+12$

\soln

\solnsteps

Using the distributive property organizer\\

The factors of $(x^2+7x+12)$ are $(\cRed{x}+\cBlue{3})(\cRed{x}+\cBlue{4})$

\begin{align*}
1x^2+7x+12 && \text{MId} \eqref{eq:mid1}  
\end{align*}

\begin{tikzpicture}[scale=1, auto]

% Place nodes

\node[firstterm](11){$1x$}; \node[factoradd,right=of 11](plus1){$+$}; \node[secondterm, right=of plus1](12){$3$};
\node[firstterm, below=of 11](21){$1x$}; \node[factoradd,right=of 21](plus2){$+$}; \node[secondterm, right=of plus2](22){$4$};

\node[multiply, below=of 21](31){$1x^2$};
\node[multiply, below=of 22](32){$7x$};

\node[multiply, right=of 12](13){$3x$};
\node[multiply, right=of 22](23){$4x$};

\node[add, below=of 23](33){$12$};

\path [line](11) edge[bend right=30]node[color=black, midway, left]{$\times$}(21);
\path [line](12) edge[bend left=30]node[color=black, midway, right]{$\times$}(22);
\path [line](21)--(31);

\path [line](21) edge[bend left=30]node[color=black, pos=0.37, below]{$\times$}(12);
\path [line](11) edge[bend right=30](22);
\path [line](22)--(32);

\path [line](12)--(13);
\path [line](22)--(23);

\path [line](13)--node[color=black, midway, right]{$+$}(23);
\path [line](23)--(33);

\end{tikzpicture}

\begin{align*}
(1x+3)(1x+4)\\
(x+3)(x+4) && \text{MId} \eqref{eq:mid2} 
\end{align*}
\end{example}


%-=-=-= EXAMPLE
\begin{example}[id:20151016-063338] \label{20151016-063338}\index{Example!20151016-063338} \hfill \\

Simplify by factoring $x^2+x-2$

\soln

\solnsteps

Using the distributive property organizer\\

The factors of $(x^2+x-2)$ are $(\cRed{x}+\cBlue{2})(\cRed{x}-\cBlue{1})$\\

\begin{align*}
1x^2+1x-2 && \text{AId} \eqref{eq:aid1} \\
1x^2+1x+\neg 2 && \text{DOS} \eqref{eq:dos1} \\
\end{align*}

\begin{tikzpicture}[scale=1, auto]

% Place nodes

\node[firstterm](11){$1x$}; \node[factoradd,right=of 11](plus1){$+$}; \node[secondterm, right=of plus1](12){$2$};
\node[firstterm, below=of 11](21){$1x$}; \node[factoradd,right=of 21](plus2){$+$}; \node[secondterm, right=of plus2](22){$-1$};

\node[multiply, below=of 21](31){$1x^2$};
\node[multiply, below=of 22](32){$\neg 2$};

\node[multiply, right=of 12](13){$2x$};
\node[multiply, right=of 22](23){$\neg 1 x$};

\node[add, below=of 23](33){$1x$};

\path [line](11) edge[bend right=30]node[color=black, midway, left]{$\times$}(21);
\path [line](12) edge[bend left=30]node[color=black, midway, right]{$\times$}(22);
\path [line](21)--(31);

\path [line](21) edge[bend left=30]node[color=black, pos=0.37, below]{$\times$}(12);
\path [line](11) edge[bend right=30](22);
\path [line](22)--(32);

\path [line](12)--(13);
\path [line](22)--(23);

\path [line](13)--node[color=black, midway, right]{$+$}(23);
\path [line](23)--(33);

\end{tikzpicture}

\begin{align*}
(1x+2)(1x + \neg 1) \\
(1x+2)(1x-1) && \text{DOS} \eqref{eq:dos2} \\
(x+2)(x-1) && \text{MId} \eqref{eq:mid2}  
\end{align*}


\end{example}

\end{document}

