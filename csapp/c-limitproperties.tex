\documentclass[20150903-160354-rs2.2-MarksMathNotebook.tex]{subfiles}

\begin{document}
%-=-=-=-=-=-=-=-=-=-=-=-=-=-=-=-=-=-=-=-=-=-=-=-=
%
%	CHAPTER
%
%-=-=-=-=-=-=-=-=-=-=-=-=-=-=-=-=-=-=-=-=-=-=-=-=

\chapter{Limit Properties}

%-=-=-=-=-=-=-=-=-=-=-=-=-=-=-=-=-=-=-=-=-=-=-=-=
%	SECTION:
%-=-=-=-=-=-=-=-=-=-=-=-=-=-=-=-=-=-=-=-=-=-=-=-=
\section{Algebraic Limit Theorem}\index{Algebraic Limit Theorem}

%-=-=-= RULE
\begin{arule}[Algebraic Limit Theorem of a Constant(ALTC)]\index{AlgebraicLimitTheorem! Limit of a Constant}
If $g(x)=a$, where $A$ is a constant, then
\begin{align}
	\displaystyle \lim_{x \to c} \left[ A \right] &= A \label{eq:altc}
\end{align}
\end{arule}

%-=-=-= RULE
\begin{arule}[Algebraic Limit Theorem of a Sum (ALTS)]\index{AlgebraicLimitTheorem! Limit of a Sum}
If both the limits $\displaystyle \lim_{x \to c} g(x)=L_1$ and $\displaystyle \lim_{x \to c} h(x)=L_2$ exist, then
\begin{align}
	\displaystyle \lim_{x \to c} \left[ g(x)+h(x) \right] &= \displaystyle \lim_{x \to c} g(x)+ \displaystyle \lim_{x \to c} h(x) \label{eq:alts} 
\end{align}
\end{arule}

%-=-=-= RULE
\begin{arule}[Algebraic Limit Theorem of a Difference (ALTD)]\index{AlgebraicLimitTheorem! Limit of a Difference}
If both the limits $\displaystyle \lim_{x \to c} g(x)=L_1$ and $\displaystyle \lim_{x \to c} h(x)=L_2$ exist, then
\begin{align}
	\displaystyle \lim_{x \to c} \left[ g(x)-h(x) \right] &= \displaystyle \lim_{x \to c} g(x)- \displaystyle \lim_{x \to c} h(x) \label{eq:alts} 
\end{align}
\end{arule}

%-=-=-= RULE
\begin{arule}[Algebraic Limit Theorem of a Product (ALTPr)]\index{AlgebraicLimitTheorem! Limit of a Product}
If both the limits $\displaystyle \lim_{x \to c} g(x)=L_1$ and $\displaystyle \lim_{x \to c} h(x)=L_2$ exist , then
\begin{align}
	\displaystyle \lim_{x \to c} \left[ g(x) \cdot h(x) \right] &= \displaystyle \lim_{x \to c} g(x) \cdot \displaystyle \lim_{x \to c} h(x) \label{eq:altpr} 
\end{align}
\end{arule}

%-=-=-= RULE
\begin{arule}[Algebraic Limit Theorem of a Quotient (ALTQ)]\index{AlgebraicLimitTheorem! Limit of a Quotient}
If both the limits $\displaystyle \lim_{x \to c} g(x)=L_1$ and $\displaystyle \lim_{x \to c} h(x)=L_2$ exist and $L_2 \ne 0$ , then
\begin{align}
	\displaystyle \lim_{x \to c} \left[ \dfrac{g(x)}{h(x)}\right] &=  \dfrac{\displaystyle \lim_{x \to c} g(x)}{\displaystyle \lim_{x \to c} h(x)}  \label{eq:altq} 
\end{align}
\end{arule}


\end{document}