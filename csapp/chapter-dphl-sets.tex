\documentclass[20150903-160354-rs2.2-MarksMathNotebook.tex]{subfiles}

\begin{document}
%-=-=-=-=-=-=-=-=-=-=-=-=-=-=-=-=-=-=-=-=-=-=-=-=
%
%	CHAPTER
%
%-=-=-=-=-=-=-=-=-=-=-=-=-=-=-=-=-=-=-=-=-=-=-=-=

\chapterimage{Pictures/chapter_head_2.pdf} % Chapter heading image

\chapter{Sets}

\section{Sets}

The study of sets and their properties is the object of set theory.

%-=-=-= DEFINITION

\begin{definition}[Set]\index{Set}
A \textbf{set} A set is a finite or infinite collection of objects in which order has no significance, and multiplicity is generally also ignored (\textit{unlike a list}).\\

Members of a set are often referred to as elements and the notation $a \in A$ is used to denote that $a$ is an element of a set $A$.
\end{definition}

\[
\underbrace{S}_{\text{Set Name}}=\{\underbrace{a, b, c}_{\text{elements}}\}
\]

\begin{remark}
We are intentionally using a non-specific definition of a set.
\end{remark}

\begin{definition}[Cardinality]\index{Cardinality}
The number of distinct elements of a set \(S\) is called its \textbf{cardinality}, which is represented as \(\left|S\right|\)
\end{definition}


\begin{definition}[Subset]\index{Subset}
Suppose that $A$ and $B$ are sets.  We say that $A$ is a \textbf{subset} of $B$ if whenever $x \in A$, then $x \in B$.

\[
A \subseteq B
\]

We could also say that all the elements of set $A$ are also elements of set $B$.
\end{definition}

\begin{definition}[Proper Subset]\index{Proper Subset}
Suppose that $A$ and $B$ are sets.  We say that $A$ is a \textbf{\textit{proper} subset} of $B$ if whenever $x \in A$, then $x \in B$, but $A \ne B$

\[
A \subset B
\]

\end{definition}

\begin{remark}
Some textbooks makes a point of differentiating between a subset and a proper subset.  This convention may later cause trouble, in terms of notation, as different writers may use the notation $A \subset B$ to denote a subsets. Making matters worse $A \subset B$ can sometimes be used to describe a proper subset as does the textbook.
\end{remark}

\begin{definition}[Equality of Sets]\index{Equality of Sets}
Suppose that both $A \subseteq B$ and $B \subseteq A$, we then say $A$ and $B$ are equal and write

\[
A=B
\]
\end{definition}

\begin{definition}[Empty Set]\index{Empty Set}
The empty set has no elements.  It is written $\emptyset$ and called the \textbf{empty set} or \textbf{null set}.

\[
\emptyset = \{\,\}
\]
\end{definition}

\begin{remark}
$\emptyset$ and $\set{\emptyset}$ are not the same thing.  By definition $\emptyset$ has no elements; however, $\set{\emptyset}$ is a set with one element, which happens to be the empty set.
\end{remark}

\section{Well-Known Sets}

\begin{itemize}
    \item Naturals: $\mathbb{N}= \set{1, 2, 3, 4, \ldots }$
    \item Integers: $\mathbb{Z}= \set{\ldots, -3, -2, -1, 0, 1, 2, 3, \ldots }$
    \item Rationals: $\mathbb{Q}= \set{a/b \suchthat a \in \mathbb{Z}, b \in \mathbb{Z}, b \ne 0 } $
\end{itemize}

\subsection{Rationals}

\begin{definition}[Field]{Field}
A \textbf{field} $F$ is a set of elements that contains at least the elements 0 and 1, given the operations of addition and multiplication, and satisfies the field axioms.
\end{definition}

Field Axioms

\begin{description}
\item[addition is commutative] $a+b=b+a \; \forall a,b,c \in F$
\item[addition is associative] $(a+b)+c = a+(b+c) \; \forall a,b \in F$
\item[additive identity] $\exists 0 \in F$ such that $a+0=a \; \forall a \in F$
\item[additive inverse] $\forall a \in F \; \exists -a \in F$ such that $-a+a=0$
\item[multiplication is commutative] $a\times b=b\times a \; \forall a,b \in F$
\item[multiplication is associative] $(a\times b) \times c = a\times (b \times c) \; \forall a,b,c \in F$
\item[multiplicative identity] $\exists 1 \in F$ such that $a \times 1 = a \; \forall a \in F$
\item[multiplicative inverse] $\forall a \in F \setminus \set{0} \exists a^{-1} \in F$ such that $a^{-1} \times a = 1$
\item[distributive law] $a \times (b +c) = a \times b + a \times c \; \forall a, b, c \in F$
\end{description}

The usual priority to $\times$ over $+$ to reduce the number of delimiters.

\section{Set Operations}

\begin{definition}[Union]\index{Union}
\[
A \cup B = \set{x \suchthat x \in a \text{ or } x \in B }
\]
\end{definition}

\begin{remark}
The word \textit{or} is not exclusive meaning that we have $x \in A \cup B$ even if $x$ is an element of both set $A$ and set $B$.
\end{remark}

\begin{definition}[Intersection]\index{Intersection}
\[
A \cap B = \set{x \suchthat x \in a \text{ and } x \in B}
\]
\end{definition}

\begin{definition}{Disjoint Sets}\index{Disjoint Sets}
We say sets $A$ and $B$ are disjoint when

\[
A \cap B = \emptyset
\]

for all sets $A$ and $B$
\end{definition}

\begin{property}[Commutative Law of Union]\index{Commutative Law of Union}
\[
A \cup B = B \cup A
\]
for all sets $A$ and $B$
\end{property}

\begin{property}[Commutative Law of Intersection]\index{Commutative Law of Intersection}
\[
A \cap B = B \cap A
\]
for all sets $A$ and $B$
\end{property}

\begin{property}[Distributive Law of Intersection over Union]\index{Distributive Law of Intersection over Union}

\[
A \cap (B \cup C) = (A \cap B) \cup (A \cap C)
\]

\end{property}

\begin{property}[Distributive Law of Intersection over Union]\index{Distributive Law of Intersection over Union}

\[
A \cup (B \cap C) = (A \cup B) \cap (A \cup C)
\]

\end{property}

\begin{property}[Union of a Set and the Empty Set]\index{Union of a Set and the Empty Set}

\[
A \cup \emptyset = A
\]
For set A
\end{property}

\begin{property}[Intersection of a Set and the Empty Set]\index{Intersection of a Set and the Empty Set}

\[
A \cap \emptyset = \emptyset
\]
For set A
\end{property}

\begin{definition}[Difference of two sets]\index{Difference of two sets}
\[
A \setminus B \set{x \suchthat x \in A, x \notin B}
\]

For all sets A and B
\end{definition}

\begin{definition}[Universe Set]\index{Universe Set}
The universe set $U$ is a given fixed set from which all subsets are discussed from.
\end{definition}

\begin{definition}[Compliment Set]\index{Compliment Set}
If $A$ is a set such that $A \subseteq B$, then

\[
A'=U \setminus A
\]

or

\[
A'=\set{x \suchthat x \notin A}
\]

where $A'$ is called the \textbf{compliment} of A.
\end{definition}

\begin{remark}
Complimentation has no meaning unless there is a universe $U$.
\end{remark}

\begin{property}{Double Compliment}\index{Double Compliment}
\[
A''=A
\]
\end{property}

\begin{property}
\[
A \cup A' = U
\]
\end{property}

\begin{property}
\[
A \cap A' = \emptyset
\]
\end{property}

\end{document}

