\documentclass[20150903-160354-rs2.2-MarksMathNotebook.tex]{subfiles}

\begin{document}
%-=-=-=-=-=-=-=-=-=-=-=-=-=-=-=-=-=-=-=-=-=-=-=-=
%
%	CHAPTER
%
%-=-=-=-=-=-=-=-=-=-=-=-=-=-=-=-=-=-=-=-=-=-=-=-=

\chapter{Curve Sketching}

\subsection{Finding the vertex of a quadratic function using differentiation.}

We can find the vertex of a quadratic function, $f(x)$ using differentiation by:

\begin{enumerate}
\item Differentiate the function: Find $f'(x)$.
\item Set the derivative equal to zero: $f'(x)=0$.
\item Find the abscissa of the vertex by solving the equation $f'(x)=0$ for $x$ to find the critical $x$ value: $x=k$.
\item Find the ordinate of the vertex by substituting the value of critical value $x=k$ into the function $f(x)$: Evaluate $f(k)$
\end{enumerate}

%-=-=-= EXAMPLE
\begin{example}[id:20151008-110208] \label{20151008-110208}\index{Example!20151008-110208} \hfill \\
Find the vertex of the quadratic function, $f(x)=ax^2+bx+c$, using differentiation.

\soln

\solnsteps

1. Find the derivative of $f(x)$
\begin{align*}
\farg{f(x)}' &= \farg{ax^2+bx+c}' && \text{SPE} \eqref{eq:spe} \\
f'(x) &= \farg{ax^2}' + \farg{bx}' + \farg{c}' && \text{DS} \eqref{eq:ds1} \\
f'(x) &= a\farg{x^2}' + b \farg{x}' + \farg{c}' && \text{DCM} \eqref{eq:dcm1} \\
f'(x) &= 2 \cdot a \cdot{x} + b \cdot 1 + \farg{c}' && \text{DPo} \eqref{eq:dpo1} \\
f'(x) &= 2 \cdot a \cdot{x} + 1 \cdot b + \farg{c}' && \text{CPM} \eqref{eq:cpm} \\
f'(x) &= 2ax + 1b+ \farg{c}' && \text{CTJ} \eqref{eq:ctj} \\
f'(x) &= 2ax + b + \farg{c}' && \text{MId} \eqref{eq:mid2} \\
f'(x) &= 2ax + b + 0 && \text{DC} \eqref{eq:dc1} \\
f'(x) &= 2ax + b && \text{AId} \eqref{eq:aid2} \\
\end{align*}

2. Set the derivative equal to zero and solve for $x$.

\begin{align*}
f'(x) & = 0 \\
2ax + b &= 0  \\
x &= -\frac{b}{2a}  \text{\, goto \,} \, \ref{20151015-104754}
\end{align*}

The abscissa of the vertex is $x=-\frac{b}{2a}$.

3.  Find the ordinate of the vertex by substituting the argument $x=-\frac{b}{2a}$ into $f(x)$

%\begin{align*}
%f(x) & = ax^2+bx+c \\
%f\left(-\frac{b}{2a}\right) &= a
%\end{align*}
\end{example}

%-=-=-= EXAMPLE
\begin{example}[id:20150923-152515] \label{20150923-152515}\index{Example!20150923-152515} \hfill \\
Find the vertex of the parabola $y=x^2-2x-6$ using differentiation.
\soln
\solnsteps
1. Differentiate the function.
\begin{align*}
f(x) &=x^2-2x-6 \\
f(x) &=x^2+ \neg 2x + \neg 6 && \text{DOS} \eqref{eq:dos1} \\
\farg{f(x)}' &= \farg{x^2+ \neg 2x + \neg 6}' && \text{SPE} \eqref{eq:spe} \\
f'(x) &= \farg{x^2}'+ \farg{\neg 2x}' + \farg{\neg 6}' && \text{DS} \eqref{eq:ds1} \\
f'(x) &= \farg{x^2}'+ \neg 2 \farg{x}' + \farg{\neg 6}' && \text{DCM} \eqref{eq:dcm1} \\
f'(x) &= 2x + \neg 2 + \farg{\neg 6}' && \text{DPo} \eqref{eq:dpo1} \\
f'(x) &= 2x + \neg 2 + 0 && \text{DC} \eqref{eq:dc1} \\
f'(x) &= 2x + \neg 2 && \text{AId} \eqref{eq:aid2} \\
f'(x) &= 2x - 2 && \text{DOS} \eqref{eq:dos2}
\end{align*}

2 and 3. Set the derivative equal to zero and solve for $x$

\begin{align*}
2x-2 & = 0\\
x &= 1
\end{align*}

4. Find the value of $f(1)$

\begin{align*}
f(x) & = x^2-2x-6 \\
f(1) &= \farg{1}^2-2\farg{1}-6 && \text{SPE} \eqref{eq:spe} \\
f(1) &= -7  &&\text{Evaluate}
\end{align*}

The vertex of this parabola is the point $(1,-7)$
\end{example}




\end{document}

