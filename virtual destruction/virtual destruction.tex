%%%%%%%%%%%%%%%%%%%%%%%%%%%%%%%%%%%%%%%%%%%%%%%%%%%%%%%%%%%% 
% Name: XeTeX+xeCJK日常使用模板
% Author: Lox Freeman
% Email: xiaohanyu1988@gmail.com
% 、
% 本文档可以自由转载、修改,希望能给广大TeXer的中文之路提供一些方便。
%%%%%%%%%%%%%%%%%%%%%%%%%%%%%%%%%%%%%%%%%%%%%%%%%%%%%%%%%%%% 

\documentclass[a4paper, 11pt, titlepage]{article}

%%%%%%%%%%%%%%%%%%%%%%%%% xeCJK相关宏包%%%%%%%%%%%%%%%%%%%%%%%%%
\usepackage{xltxtra,fontspec,xunicode}

\usepackage[slantfont, boldfont, CJKchecksingle]{xeCJK}
\CJKsetecglue{\hskip 0.15em plus 0.05em minus 0.05em}
% slanfont: 允许斜体
% boldfont: 允许粗体
% CJKnormalspaces: 仅忽略汉字之间的空白,但保留中英文之间的空白。
% CJKchecksingle: 避免单个汉字单独占一行。
% CJKaddspaces: [备选]忽略汉字之间的空白,并且自动在中英文转换时插入空白。

%\CJKlanguage{zh-cn}                 % 中文标点特殊处理
\XeTeXlinebreaklocale "zh"           % 针对中文进行断行
\XeTeXlinebreakskip = 0pt plus 1pt minus 0.1pt
% 给予TeX断行一定自由度
%%%%%%%%%%%%%%%%%%%%%%%%% xeCJK%%%%%%%%%%%%%%%%%%%%%%%%%%%%%%%%

%%%%%%%%%%%%% 日常所用宏包、通通放在一起%%%%%%%%%%%%%%%%%%%%%%%%%%%%
% 什么常用的宏包都可以放这里。下面是我常用的宏包,每个都给出了简要注释
\usepackage[top=2.5cm, bottom=3cm, left=2.5cm, right=2.5cm]{geometry}
% 控制页边距
\usepackage{enumerate}               % 控制项目列表
\usepackage{multicol}                % 多栏显示

\usepackage[%
pdfstartview=FitH,%
CJKbookmarks=true,%
bookmarks=true,%
bookmarksnumbered=true,%
bookmarksopen=true,%
colorlinks=true,%
citecolor=blue,%
linkcolor=blue,%
anchorcolor=green,%
urlcolor=blue%
]{hyperref}

% \usepackage[margin=12mm]{geometry}

\usepackage{titlesec}                % 控制标题
\usepackage{titletoc}                % 控制目录
\usepackage{type1cm}                 % 控制字体大小
\usepackage{indentfirst}             % 首行缩进,用\noindent取消某段缩进
\usepackage{bbding}                  % 一些特殊符号
\usepackage{cite}                    % 支持引用
\usepackage{framed,color,xcolor}            % 支持彩色文本、底色、文本框等
\usepackage{latexsym}                % LaTeX一些特殊符号宏包
\usepackage{amsmath}                 % AMS LaTeX宏包
\usepackage{bm}                      % 数学公式中的黑斜体
\usepackage{relsize}                 % 调整公式字体大小:\mathsmaller, \mathlarger
\usepackage{soul}                    % 下划线自动回车换行
\usepackage{attachfile}              % 添加附见使用的宏包
\usepackage{parcolumns}              % 列排版
% \usepackage{setspace}
\makeindex                           % 生成索引

\makeatletter
\let\std@footnotetext\@footnotetext
\usepackage{setspace}
\let\@footnotetext\std@footnotetext
\makeatother

%%%%%%%%%%%%%%%%%%%%%%%%% 基本插图方法%%%%%%%%%%%%%%%%%%%%%%%%%%%
\usepackage{graphicx}                % 图形宏包
% \begin{figure}[htbp]               % 控制插图位置
%   \setlength{\abovecaptionskip}{0pt}    
%   \setlength{\belowcaptionskip}{10pt}
%   控制图形和上下文的距离
%   \centering                       % 使图形居中显示
%   \includegraphics[width=0.8\textwidth]{CTeXLive2008.jpg}
%   控制图形显示宽度为0.8\textwidth
%   \caption{CTeXLive2008安装过程} \label{fig:CTeXLive2008}
%   图形题目和交叉引用标签
% \end{figure}
%%%%%%%%%%%%%%%%%%%%%%%%% 插图方法结束%%%%%%%%%%%%%%%%%%%%%%%%%%%

%%%%%%%%%%%%%%%%%%%%%%%%% 绘图方法%%%%%%%%%%%%%%%%%%%%%%%%%%%
\usepackage{tikz}
\usepackage{amsmath,bm,times}
\usepackage{verbatim}

\usepackage{tabularx}
% \usepackage{hhline}
% \usepackage{colortbl}
\usetikzlibrary{shapes,arrows,shadows,fit,snakes}
\usetikzlibrary{calc}           %coordinate

\newlength\Textwd
\setlength\Textwd{3cm}
\newcommand\Textbox[2]{%
  \parbox[c][#1][c]{\Textwd}{\linespread{0.5}\centering#2}}

\makeatletter
\DeclareRobustCommand{\rvdots}{%
  \vbox{
    \baselineskip6\p@\lineskiplimit\z@
    \kern-\p@
    \hbox{.}\hbox{.}\hbox{.}
  }}
\makeatother

\ifx\du\undefined
  \newlength{\du}
\fi
\setlength{\du}{15\unitlength}
%%%%%%%%%%%%%%%%%%%%%%%%% 绘图方法结束%%%%%%%%%%%%%%%%%%%%%%%%%%%

%%%%%%%%%%%%%%%%%%%%%%%%% fancyhdr设置页眉页脚%%%%%%%%%%%%%%%%%%%%
\usepackage{fancyhdr}                % 页眉页脚
\pagestyle{fancy}                    % 页眉页脚风格
\setlength{\headheight}{15pt}        % 有时会出现\headheight too small的warning
% \fancyhf{}                          % 清空当前页眉页脚的默认设置
%%%%%%%%%%%%%%%%%%%%%%%%% fancyhdr设置结束%%%%%%%%%%%%%%%%%%%%%%%

%%%%%%%%%%%%%%%%%%%%%%%%% xeCJK字体设置%%%%%%%%%%%%%%%%%%%%%%%%%
\punctstyle{kaiming}                                        % 设置中文标点样式
% 支持全角、半角、kaiming等多种方式
% \setmainfont[Mapping=tex-text]{Adobe Garamond Pro}
% \newfontfamily\courier{Courier New}
\setCJKmainfont[BoldFont={Microsoft YaHei:style=Bold},ItalicFont={Microsoft YaHei}]{YouYuan}
\setCJKsansfont{Adobe Heiti Std}
\setCJKmonofont{Adobe Fangsong Std}
%Microsoft YaHei
%YouYuan
%%%% 定义新字体%%%%
\setCJKfamilyfont{song}{Adobe Song Std}
\setCJKfamilyfont{kai}{Adobe Kaiti Std}
\setCJKfamilyfont{hei}{Adobe Heiti Std}
\setCJKfamilyfont{fangsong}{Adobe Fangsong Std}
\setCJKfamilyfont{lisu}{LiSu}
\setCJKfamilyfont{youyuan}{YouYuan}

\newcommand{\song}{\CJKfamily{song}}                       % 自定义宋体
\newcommand{\kai}{\CJKfamily{kai}}                         % 自定义楷体
\newcommand{\hei}{\CJKfamily{hei}}                         % 自定义黑体
\newcommand{\fangsong}{\CJKfamily{fangsong}}               % 自定义仿宋体
\newcommand{\lisu}{\CJKfamily{lisu}}                       % 自定义隶书
\newcommand{\youyuan}{\CJKfamily{youyuan}}                 % 自定义幼圆
%%%%%%%%%%%%%%%%%%%%%%%%% xeCJK字体设置结束%%%%%%%%%%%%%%%%%%%%%%


\usepackage[framemethod=tikz]{mdframed}
\newtheorem{question}{Question}
\mdfdefinestyle{que}{
  linecolor=cyan,
  backgroundcolor=cyan!20,
}
\surroundwithmdframed[style=que]{question}

%%%%%%%%%%%%%%%%%%%%%%%%% listings宏包粘贴源码%%%%%%%%%%%%%%%%%%%%
\usepackage{listings}                % 方便粘贴源代码,部分代码高亮功能
\lstloadlanguages{}                  % 所要粘贴代码的编程语言

\include{rgb}

\newfontfamily\listingsfont[Scale=0.9]{YaHei Consolas Hybrid}
\newfontfamily\listingsfontinline[Scale=0.9]{YaHei Consolas Hybrid}

\definecolor{sh_keyword}{rgb}{0.06, 0.10, 0.98}   % #101AF9
%\definecolor{orange}{RGB}{255,127,0}
\definecolor{shadecolor}{rgb}{0.83, 0.83, 0.83}
%\definecolor{shadecolor}{RGB}{236, 236, 236}

\definecolor{monokai-grey-dark}{RGB}{39, 40, 34}   % #272822
\definecolor{monokai-yellow-light}{RGB}{248, 248, 246}   % #F8F8F2

\definecolor{monokai-green}{RGB}{166, 226, 42}

\def\lstsmallmath{\leavevmode\ifmmode \scriptstyle \else  \fi}
\def\lstsmallmathend{\leavevmode\ifmmode  \else  \fi}

\renewcommand{\lstlistingname}{代码}

\lstset {
  language=c++,
  backgroundcolor=\color{monokai-grey-dark},
%  frame=shadowbox,
  % breaklines,
  rulesepcolor=\color{red!20!green!20!blue!20},
  showspaces=false,showtabs=false,tabsize=4,
  numberstyle=\tiny\color{black},numbers=left,
  basicstyle= \listingsfont\color{monokai-yellow-light},
  stringstyle=\color{dark green},
  keywordstyle = \color{monokai-green}\bfseries,
  % commentstyle=\tiny\color{forest green}\itshape,
  commentstyle=\color{forest green}\itshape,
  captionpos=b,
  showspaces=false,showtabs=false, showstringspaces=false,
  xleftmargin=0.7cm, xrightmargin=0.5cm,
  lineskip=-0.3em,
  breaklines=tr,
  escapebegin={\lstsmallmath}, escapeend={\lstsmallmathend},
  extendedchars=false
}

\lstnewenvironment{acol}[1][]{\lstset{language={[x86masm]Assembler},#1}}{}
\newenvironment{parcolumenv}[1] {\begin{spacing}{#1}}{\end{spacing}}

%%%%%%%%%%%%%%%%%%%%%%%%% listings宏包设置结束%%%%%%%%%%%%%%%%%%%%


%%%%%%%%%%%%%%%%%%%%%%%%% 一些关于中文文档的重定义%%%%%%%%%%%%%%%%%

%%%% 数学公式定理的重定义%%%%
\newtheorem{example}{例}                                   % 整体编号
\newtheorem{algorithm}{算法}
\newtheorem{theorem}{定理}[section]                        % 按 section 编号
\newtheorem{definition}{定义}
\newtheorem{axiom}{公理}
\newtheorem{property}{性质}
\newtheorem{proposition}{命题}
\newtheorem{lemma}{引理}
\newtheorem{corollary}{推论}
\newtheorem{remark}{注解}
\newtheorem{condition}{条件}
\newtheorem{conclusion}{结论}
\newtheorem{assumption}{假设}

%%%% 章节等名称重定义%%%%
\renewcommand{\contentsname}{目录}     
\renewcommand{\indexname}{索引}
\renewcommand{\listfigurename}{插图目录}
\renewcommand{\listtablename}{表格目录}
\renewcommand{\figurename}{图}
\renewcommand{\tablename}{表}
\renewcommand{\appendixname}{附录}

%%%% 设置chapter、section与subsection的格式%%%%
\titleformat{\chapter}{\centering\huge}{第\thechapter{}章}{1em}{\textbf}
\titleformat{\section}{\centering\LARGE}{\thesection}{1em}{\textbf}
\titleformat{\subsection}{\Large}{\thesubsection}{1em}{\textbf}
%%%%%%%%%%%%%%%%%%%%%%%%% 中文重定义结束%%%%%%%%%%%%%%%%%%%%


%%%%%%%%%%%%%%%%%%%%%%%%% 一些个性设置%%%%%%%%%%%%%%%%%%%%%%
% \renewcommand{\baselinestretch}{1.3}     % 效果同\linespread{1.3}
% \pagenumbering{arabic}                   % 设定页码方式,包括arabic、roman等方式
% \sloppy                                  % 有时LaTeX无从断行,产生overfull的错误,
% 这条命令降低LaTeX断行标准
\setlength{\parskip}{0.5\baselineskip}     % 设定段间距
\linespread{1.60}                           % 设定行距
\newcommand{\pozhehao}{\kern0.3ex\rule[0.8ex]{2em}{0.1ex}\kern0.3ex}
% 中文破折号,据说来自清华模板

\newcommand*{\TitleFont}{\usefont{\encodingdefault}{\rmdefault}{b}{n}\fontsize{32}{40}\selectfont}
% 标题字体设置

\renewcommand{\today}{\number\year 年 \number\month 月 \number\day 日}

%%%%%%%%%%%%%%%%%%%%%%%%% 个性设置结束%%%%%%%%%%%%%%%%%%%%%%


%%%%%%%%%%%%%%%%%%%%%%%%% 正文部分%%%%%%%%%%%%%%%%%%%%%%%%%
\begin{document}
\setlength{\parindent}{2em}
% 设定首行缩进为2em。注意此设置一定要在document环境之中。
% 这可能与\setlength作用范围相关

\title{\TitleFont 一个内存泄露问题的定位}
\author{\href{mailto:qinxinliang@huawei.com}{秦新良}}
\date{\today}

\maketitle

\tableofcontents
\newpage

\section{问题现象}
USCDBV100R007C10版本TR5在即,性能、稳定性、安全测试全面启动,测试部的一套环境上连续业务呼叫一整天后,出现网元下的所有PGW进程从凌晨2点到5点陆续全部重启。

\section{问题定位}
\subsection{初步分析}
通过分析进程退出时产生的黑匣子发现,重启是因为当时PGW进程的虚拟内存已达到3G,后续的业务处理由于内存申请失败而导致进程异常退出。

重启的原因已非常明显,内存泄露导致。但是是哪种业务场景导致的内存泄露呢?这是整个定位过程中的重中之重。

\subsection{问题重现}
出现问题的环境是BEC组网的环境,业务运行在SP上,跑的是UPA的指令。这些指令的特殊性在于,客户端过来的一条UPA指令,会被PGW通过三方适配拆成一条UPA指令和一条HLR指令,分别发给远端UPA和HLR的SP上执行。除此之外,该环境上的全局数据刷新开关是打开的,PGW每隔一小时刷新一次全局数据。

通过分析系统每分钟的top日志发现,业务进程的虚拟内存每隔一小时上涨一次,与全局数据刷新的周期相吻合,所以重现及排查问题的重点就放在了全局数据刷新上\footnote{还有一个原因是全局数据刷新在历史的版本中就有内存相关的疑难杂症,在171版本刚刚解决,而且在171版本中又加了一个按FE刷新,所以它的疑点最多。}。

不幸的是,按照全局数据刷新这个思路来重现问题,问题一直没有重现。但由于该场景是重点怀疑对象,所以我们几个人还是想尽各种办法来排查、构造全局数据的异常场景,期待问题的重现。

经过了一天一夜后,问题还是没能稳定的重现。虽然运行过程中,进程的内存会有一些浮动,但整体来看是一个稳定的状态。最后没办法,只能换个方向,通过跑UPA的业务指令试着重现问题。

持续执行UPA的命令,发现内存是有一些增涨,但非常缓慢,而且也不是持续增涨。如果要等到这种场景把内存泄露完来确认问题,估计一天又要过去了。但看到内存有一些增涨,总归是给定位问题提供了一个新的方向,先只能按着这个方向继续往下定位。

\subsection{问题定位}
在重现问题的过程中,查了一些tcmalloc内存管理相关资料,发现tcmalloc在申请内存的时候,是一次申请大片内存,后续应用程序有内存申请的时候,直接从这块大内存中分配,如果当前申请的内存不够的话,才会向系统申请内存。通过分析malloc\_stat的结果,验证了这一结论。既然这样,那如果是UPA的指令存在内存泄露,外在表现确实不会是连续的内存增涨,而是伴随着tcmalloc已申请的内存不足时再次向系统申请时才会出现内存上涨。有了这个分析结论后,问题重现及排查的重点就放在了UPA指令上。

最后通过\href{http://valgrind.org/}{valgrind}将PGW拉起并执行UPA的指令,运行一段时间后,将进程退出,生成valgrind的报告如下:
\lstinputlisting{valgrind.txt}

从报告中的{\color{blue}LEAK SUMMARY}可以看出,进程中存在38个字节(definitely\ lost)的内存泄露。继续分析报告,报告中指出,函数mem\_leak\_test()中存在内存泄露。函数mem\_leak\_test()的定义如代码\footnote{这段代码只是用来说明问题,实际的业务逻辑比这复杂的多,而且内存也不是在同一函数内申请和释放。}\ref{lst1}所示。
\lstinputlisting[label=lst1,caption=示例代码]{C1.cpp}

从代码\ref{lst1}中可以看出,在函数mem\_leak\_test()中,申明了基类Base的指针pObj,其指向的是派生类D1的对象,然后通过pObj将动态申请的对象释放。

接着再看类Base和D1的定义,发现基类Base的析构函数未显示定义,那么编译器会自动为Base生成默认的析构函数,但默认生成的析构是非virtual的,这就导致了通过基类指针pObj释放派生类的对象时,派生类对象的析构函数不会被调用,如果派生类中有动态的内存申请的话,就会出现内存泄露。具体来说就是,D1的析构函数不会被调用,那么D1的成员m\_strData2的析构函数就不会被调用,最终导致m\_strData2里动态申请的内存得不到释放。

\section{修改方案}
修改方案很简单,将基类的析构函数定义为虚函数即可,如代码\ref{lst2}所示。
\lstinputlisting[label=lst2,caption=修改后的代码]{C2.cpp}

基类Base的析构函数申明为虚函数后,Base和D1的内存布局如图\ref{fig1}所示。
\begin{figure}[!htb]
  \setlength{\abovecaptionskip}{0pt}
  \begin{center}
    \begin{singlespace}
      \begin{tikzpicture}
        \pgftransformxscale{1.000000}
        \pgftransformyscale{-1.000000}
        \definecolor{dialinecolor}{rgb}{0.000000, 0.000000, 0.000000}
        \pgfsetstrokecolor{dialinecolor}
        \definecolor{dialinecolor}{rgb}{1.000000, 1.000000, 1.000000}
        \pgfsetfillcolor{dialinecolor}
        \definecolor{dialinecolor}{rgb}{0.564706, 0.933333, 0.564706}
        \pgfsetfillcolor{dialinecolor}
        \fill (-12.997997\du,-9.619740\du)--(-12.997997\du,-8.147112\du)--(-8.431622\du,-8.147112\du)--(-8.431622\du,-9.619740\du)--cycle;
        \pgfsetlinewidth{0.045000\du}
        \pgfsetdash{}{0pt}
        \pgfsetdash{}{0pt}
        \pgfsetmiterjoin
        \definecolor{dialinecolor}{rgb}{0.000000, 0.000000, 0.000000}
        \pgfsetstrokecolor{dialinecolor}
        \draw (-12.997997\du,-9.619740\du)--(-12.997997\du,-8.147112\du)--(-8.431622\du,-8.147112\du)--(-8.431622\du,-9.619740\du)--cycle;
        % setfont left to latex
        \definecolor{dialinecolor}{rgb}{0.000000, 0.000000, 0.000000}
        \pgfsetstrokecolor{dialinecolor}
        \node at (-10.714809\du,-8.791109\du){\scriptsize \textbf{Base::\~{}Base()}};
        \definecolor{dialinecolor}{rgb}{0.564706, 0.933333, 0.564706}
        \pgfsetfillcolor{dialinecolor}
        \fill (-6.660274\du,-13.721040\du)--(-6.660274\du,-12.056040\du)--(-2.794549\du,-12.056040\du)--(-2.794549\du,-13.721040\du)--cycle;
        \pgfsetlinewidth{0.045000\du}
        \pgfsetdash{}{0pt}
        \pgfsetdash{}{0pt}
        \pgfsetmiterjoin
        \definecolor{dialinecolor}{rgb}{0.000000, 0.000000, 0.000000}
        \pgfsetstrokecolor{dialinecolor}
        \draw (-6.660274\du,-13.721040\du)--(-6.660274\du,-12.056040\du)--(-2.794549\du,-12.056040\du)--(-2.794549\du,-13.721040\du)--cycle;
        % setfont left to latex
        \definecolor{dialinecolor}{rgb}{0.000000, 0.000000, 0.000000}
        \pgfsetstrokecolor{dialinecolor}
        \node at (-4.727412\du,-12.765199\du){\scriptsize \textbf{vptr}};
        \pgfsetlinewidth{0.040500\du}
        \pgfsetdash{}{0pt}
        \pgfsetdash{}{0pt}
        \pgfsetmiterjoin
        \pgfsetbuttcap
        {
          \definecolor{dialinecolor}{rgb}{0.000000, 0.000000, 0.000000}
          \pgfsetfillcolor{dialinecolor}
          % was here!!!
          \pgfsetarrowsend{stealth}
          \definecolor{dialinecolor}{rgb}{0.000000, 0.000000, 0.000000}
          \pgfsetstrokecolor{dialinecolor}
          \pgfpathmoveto{\pgfpoint{-6.660274\du}{-12.888540\du}}
          \pgfpathcurveto{\pgfpoint{-7.589581\du}{-12.904826\du}}{\pgfpoint{-6.672954\du}{-9.470231\du}}{\pgfpoint{-8.431622\du}{-8.883426\du}}
          \pgfusepath{stroke}
        }
        \definecolor{dialinecolor}{rgb}{1.000000, 0.752941, 0.796078}
        \pgfsetfillcolor{dialinecolor}
        \fill (5.607986\du,-9.631980\du)--(5.607986\du,-8.159352\du)--(10.174361\du,-8.159352\du)--(10.174361\du,-9.631980\du)--cycle;
        \pgfsetlinewidth{0.045000\du}
        \pgfsetdash{}{0pt}
        \pgfsetdash{}{0pt}
        \pgfsetmiterjoin
        \definecolor{dialinecolor}{rgb}{0.000000, 0.000000, 0.000000}
        \pgfsetstrokecolor{dialinecolor}
        \draw (5.607986\du,-9.631980\du)--(5.607986\du,-8.159352\du)--(10.174361\du,-8.159352\du)--(10.174361\du,-9.631980\du)--cycle;
        % setfont left to latex
        \definecolor{dialinecolor}{rgb}{0.000000, 0.000000, 0.000000}
        \pgfsetstrokecolor{dialinecolor}
        \node at (7.891173\du,-8.803349\du){\scriptsize \textbf{D1::\~{}D1()}};
        \pgfsetlinewidth{0.040500\du}
        \pgfsetdash{}{0pt}
        \pgfsetdash{}{0pt}
        \pgfsetmiterjoin
        \pgfsetbuttcap
        {
          \definecolor{dialinecolor}{rgb}{0.000000, 0.000000, 0.000000}
          \pgfsetfillcolor{dialinecolor}
          % was here!!!
          \pgfsetarrowsend{stealth}
          \definecolor{dialinecolor}{rgb}{0.000000, 0.000000, 0.000000}
          \pgfsetstrokecolor{dialinecolor}
          \pgfpathmoveto{\pgfpoint{3.685451\du}{-12.888540\du}}
          \pgfpathcurveto{\pgfpoint{5.001866\du}{-12.830310\du}}{\pgfpoint{3.626887\du}{-9.395232\du}}{\pgfpoint{5.607986\du}{-8.895666\du}}
          \pgfusepath{stroke}
        }
        \pgfsetlinewidth{0.040500\du}
        \pgfsetdash{{\pgflinewidth}{0.200000\du}}{0cm}
        \pgfsetdash{{\pgflinewidth}{0.200000\du}}{0cm}
        \pgfsetbuttcap
        {
          \definecolor{dialinecolor}{rgb}{0.000000, 0.000000, 0.000000}
          \pgfsetfillcolor{dialinecolor}
          % was here!!!
          \definecolor{dialinecolor}{rgb}{0.000000, 0.000000, 0.000000}
          \pgfsetstrokecolor{dialinecolor}
          \draw (-2.794549\du,-13.721040\du)--(-0.180274\du,-13.721040\du);
        }
        \pgfsetlinewidth{0.040500\du}
        \pgfsetdash{{\pgflinewidth}{0.200000\du}}{0cm}
        \pgfsetdash{{\pgflinewidth}{0.200000\du}}{0cm}
        \pgfsetbuttcap
        {
          \definecolor{dialinecolor}{rgb}{0.000000, 0.000000, 0.000000}
          \pgfsetfillcolor{dialinecolor}
          % was here!!!
          \definecolor{dialinecolor}{rgb}{0.000000, 0.000000, 0.000000}
          \pgfsetstrokecolor{dialinecolor}
          \draw (-2.795050\du,-10.420876\du)--(-0.159693\du,-10.436697\du);
        }
        \definecolor{dialinecolor}{rgb}{0.117647, 0.564706, 1.000000}
        \pgfsetfillcolor{dialinecolor}
        \fill (-6.743380\du,-8.137296\du)--(-6.743380\du,-6.451943\du)--(-2.202880\du,-6.451943\du)--(-2.202880\du,-8.137296\du)--cycle;
        \pgfsetlinewidth{0.045000\du}
        \pgfsetdash{}{0pt}
        \pgfsetdash{}{0pt}
        \pgfsetmiterjoin
        \definecolor{dialinecolor}{rgb}{0.000000, 0.000000, 0.000000}
        \pgfsetstrokecolor{dialinecolor}
        \draw (-6.743380\du,-8.137296\du)--(-6.743380\du,-6.451943\du)--(-2.202880\du,-6.451943\du)--(-2.202880\du,-8.137296\du)--cycle;
        % setfont left to latex
        \definecolor{dialinecolor}{rgb}{0.000000, 0.000000, 0.000000}
        \pgfsetstrokecolor{dialinecolor}
        \node at (-4.473130\du,-7.155528\du){\scriptsize \textbf{Flat Memory}};
        \definecolor{dialinecolor}{rgb}{0.117647, 0.564706, 1.000000}
        \pgfsetfillcolor{dialinecolor}
        \fill (-0.155074\du,-7.277229\du)--(-0.155074\du,-5.591876\du)--(4.385426\du,-5.591876\du)--(4.385426\du,-7.277229\du)--cycle;
        \pgfsetlinewidth{0.045000\du}
        \pgfsetdash{}{0pt}
        \pgfsetdash{}{0pt}
        \pgfsetmiterjoin
        \definecolor{dialinecolor}{rgb}{0.000000, 0.000000, 0.000000}
        \pgfsetstrokecolor{dialinecolor}
        \draw (-0.155074\du,-7.277229\du)--(-0.155074\du,-5.591876\du)--(4.385426\du,-5.591876\du)--(4.385426\du,-7.277229\du)--cycle;
        % setfont left to latex
        \definecolor{dialinecolor}{rgb}{0.000000, 0.000000, 0.000000}
        \pgfsetstrokecolor{dialinecolor}
        \node at (2.115176\du,-6.295461\du){\scriptsize \textbf{Flat Memory}};
        \definecolor{dialinecolor}{rgb}{0.117647, 0.564706, 1.000000}
        \pgfsetfillcolor{dialinecolor}
        \fill (-0.144004\du,-4.928274\du)--(-0.144004\du,-3.242921\du)--(4.396496\du,-3.242921\du)--(4.396496\du,-4.928274\du)--cycle;
        \pgfsetlinewidth{0.045000\du}
        \pgfsetdash{}{0pt}
        \pgfsetdash{}{0pt}
        \pgfsetmiterjoin
        \definecolor{dialinecolor}{rgb}{0.000000, 0.000000, 0.000000}
        \pgfsetstrokecolor{dialinecolor}
        \draw (-0.144004\du,-4.928274\du)--(-0.144004\du,-3.242921\du)--(4.396496\du,-3.242921\du)--(4.396496\du,-4.928274\du)--cycle;
        % setfont left to latex
        \definecolor{dialinecolor}{rgb}{0.000000, 0.000000, 0.000000}
        \pgfsetstrokecolor{dialinecolor}
        \node at (2.126246\du,-3.946506\du){\scriptsize \textbf{Flat Memory}};
        \pgfsetlinewidth{0.040500\du}
        \pgfsetdash{}{0pt}
        \pgfsetdash{}{0pt}
        \pgfsetmiterjoin
        \pgfsetbuttcap
        {
          \definecolor{dialinecolor}{rgb}{0.000000, 0.000000, 0.000000}
          \pgfsetfillcolor{dialinecolor}
          % was here!!!
          \pgfsetarrowsend{stealth}
          {\pgfsetcornersarced{\pgfpoint{0.000000\du}{0.000000\du}}\definecolor{dialinecolor}{rgb}{0.000000, 0.000000, 0.000000}
            \pgfsetstrokecolor{dialinecolor}
            \draw (-2.795050\du,-11.253376\du)--(-1.587433\du,-11.253376\du)--(-1.587433\du,-7.294620\du)--(-2.202880\du,-7.294620\du);
          }}
        \pgfsetlinewidth{0.040500\du}
        \pgfsetdash{}{0pt}
        \pgfsetdash{}{0pt}
        \pgfsetmiterjoin
        \pgfsetbuttcap
        {
          \definecolor{dialinecolor}{rgb}{0.000000, 0.000000, 0.000000}
          \pgfsetfillcolor{dialinecolor}
          % was here!!!
          \pgfsetarrowsend{stealth}
          {\pgfsetcornersarced{\pgfpoint{0.000000\du}{0.000000\du}}\definecolor{dialinecolor}{rgb}{0.000000, 0.000000, 0.000000}
            \pgfsetstrokecolor{dialinecolor}
            \draw (-0.157822\du,-11.264584\du)--(-1.010524\du,-11.264584\du)--(-1.010524\du,-6.434553\du)--(-0.155074\du,-6.434553\du);
          }}
        \pgfsetlinewidth{0.040500\du}
        \pgfsetdash{}{0pt}
        \pgfsetdash{}{0pt}
        \pgfsetmiterjoin
        \pgfsetbuttcap
        {
          \definecolor{dialinecolor}{rgb}{0.000000, 0.000000, 0.000000}
          \pgfsetfillcolor{dialinecolor}
          % was here!!!
          \pgfsetarrowsend{stealth}
          {\pgfsetcornersarced{\pgfpoint{0.000000\du}{0.000000\du}}\definecolor{dialinecolor}{rgb}{0.000000, 0.000000, 0.000000}
            \pgfsetstrokecolor{dialinecolor}
            \draw (-0.159693\du,-9.604197\du)--(-0.754834\du,-9.604197\du)--(-0.754834\du,-4.085598\du)--(-0.144004\du,-4.085598\du);
          }}
        % setfont left to latex
        \definecolor{dialinecolor}{rgb}{0.000000, 0.000000, 0.000000}
        \pgfsetstrokecolor{dialinecolor}
        \node at (-4.737568\du,-14.253930\du){$\small Base$};
        % setfont left to latex
        \definecolor{dialinecolor}{rgb}{0.000000, 0.000000, 0.000000}
        \pgfsetstrokecolor{dialinecolor}
        \node at (1.780146\du,-14.156670\du){$\footnotesize D1$};
        % setfont left to latex
        \definecolor{dialinecolor}{rgb}{0.000000, 0.000000, 0.000000}
        \pgfsetstrokecolor{dialinecolor}
        \node at (7.898216\du,-10.341720\du){\tiny \textbf{virtual table for D1}};
        % setfont left to latex
        \definecolor{dialinecolor}{rgb}{0.000000, 0.000000, 0.000000}
        \pgfsetstrokecolor{dialinecolor}
        \node at (-10.709131\du,-10.382220\du){\tiny \textbf{virtual table for Base}};
        \definecolor{dialinecolor}{rgb}{1.000000, 0.752941, 0.796078}
        \pgfsetfillcolor{dialinecolor}
        \fill (-0.155093\du,-13.717072\du)--(-0.155093\du,-12.052072\du)--(3.710632\du,-12.052072\du)--(3.710632\du,-13.717072\du)--cycle;
        \pgfsetlinewidth{0.045000\du}
        \pgfsetdash{}{0pt}
        \pgfsetdash{}{0pt}
        \pgfsetmiterjoin
        \definecolor{dialinecolor}{rgb}{0.000000, 0.000000, 0.000000}
        \pgfsetstrokecolor{dialinecolor}
        \draw (-0.155093\du,-13.717072\du)--(-0.155093\du,-12.052072\du)--(3.710632\du,-12.052072\du)--(3.710632\du,-13.717072\du)--cycle;
        % setfont left to latex
        \definecolor{dialinecolor}{rgb}{0.000000, 0.000000, 0.000000}
        \pgfsetstrokecolor{dialinecolor}
        \node at (1.777770\du,-12.761231\du){\scriptsize \textbf{vptr}};
        \definecolor{dialinecolor}{rgb}{1.000000, 0.752941, 0.796078}
        \pgfsetfillcolor{dialinecolor}
        \fill (-0.157822\du,-12.097084\du)--(-0.157822\du,-10.432084\du)--(3.707903\du,-10.432084\du)--(3.707903\du,-12.097084\du)--cycle;
        \pgfsetlinewidth{0.045000\du}
        \pgfsetdash{}{0pt}
        \pgfsetdash{}{0pt}
        \pgfsetmiterjoin
        \definecolor{dialinecolor}{rgb}{0.000000, 0.000000, 0.000000}
        \pgfsetstrokecolor{dialinecolor}
        \draw (-0.157822\du,-12.097084\du)--(-0.157822\du,-10.432084\du)--(3.707903\du,-10.432084\du)--(3.707903\du,-12.097084\du)--cycle;
        % setfont left to latex
        \definecolor{dialinecolor}{rgb}{0.000000, 0.000000, 0.000000}
        \pgfsetstrokecolor{dialinecolor}
        \node at (1.775040\du,-11.141243\du){\scriptsize \textbf{m\_strData1}};
        \definecolor{dialinecolor}{rgb}{1.000000, 0.752941, 0.796078}
        \pgfsetfillcolor{dialinecolor}
        \fill (-0.159693\du,-10.436697\du)--(-0.159693\du,-8.771697\du)--(3.706032\du,-8.771697\du)--(3.706032\du,-10.436697\du)--cycle;
        \pgfsetlinewidth{0.045000\du}
        \pgfsetdash{}{0pt}
        \pgfsetdash{}{0pt}
        \pgfsetmiterjoin
        \definecolor{dialinecolor}{rgb}{0.000000, 0.000000, 0.000000}
        \pgfsetstrokecolor{dialinecolor}
        \draw (-0.159693\du,-10.436697\du)--(-0.159693\du,-8.771697\du)--(3.706032\du,-8.771697\du)--(3.706032\du,-10.436697\du)--cycle;
        % setfont left to latex
        \definecolor{dialinecolor}{rgb}{0.000000, 0.000000, 0.000000}
        \pgfsetstrokecolor{dialinecolor}
        \node at (1.773170\du,-9.480856\du){\scriptsize \textbf{m\_strData2}};
        \definecolor{dialinecolor}{rgb}{0.564706, 0.933333, 0.564706}
        \pgfsetfillcolor{dialinecolor}
        \fill (-12.996782\du,-8.149452\du)--(-12.996782\du,-6.676824\du)--(-8.430407\du,-6.676824\du)--(-8.430407\du,-8.149452\du)--cycle;
        \pgfsetlinewidth{0.045000\du}
        \pgfsetdash{}{0pt}
        \pgfsetdash{}{0pt}
        \pgfsetmiterjoin
        \definecolor{dialinecolor}{rgb}{0.000000, 0.000000, 0.000000}
        \pgfsetstrokecolor{dialinecolor}
        \draw (-12.996782\du,-8.149452\du)--(-12.996782\du,-6.676824\du)--(-8.430407\du,-6.676824\du)--(-8.430407\du,-8.149452\du)--cycle;
        % setfont left to latex
        \definecolor{dialinecolor}{rgb}{0.000000, 0.000000, 0.000000}
        \pgfsetstrokecolor{dialinecolor}
        \node at (-10.713594\du,-7.320820\du){\scriptsize \textbf{Base::\~{}Base()}};
        \definecolor{dialinecolor}{rgb}{0.564706, 0.933333, 0.564706}
        \pgfsetfillcolor{dialinecolor}
        \fill (-12.996782\du,-6.683857\du)--(-12.996782\du,-5.211229\du)--(-8.430407\du,-5.211229\du)--(-8.430407\du,-6.683857\du)--cycle;
        \pgfsetlinewidth{0.045000\du}
        \pgfsetdash{}{0pt}
        \pgfsetdash{}{0pt}
        \pgfsetmiterjoin
        \definecolor{dialinecolor}{rgb}{0.000000, 0.000000, 0.000000}
        \pgfsetstrokecolor{dialinecolor}
        \draw (-12.996782\du,-6.683857\du)--(-12.996782\du,-5.211229\du)--(-8.430407\du,-5.211229\du)--(-8.430407\du,-6.683857\du)--cycle;
        % setfont left to latex
        \definecolor{dialinecolor}{rgb}{0.000000, 0.000000, 0.000000}
        \pgfsetstrokecolor{dialinecolor}
        \node at (-10.713594\du,-5.855226\du){\scriptsize \textbf{Base::fun1()}};
        \definecolor{dialinecolor}{rgb}{0.564706, 0.933333, 0.564706}
        \pgfsetfillcolor{dialinecolor}
        \fill (-12.996782\du,-5.221387\du)--(-12.996782\du,-3.748760\du)--(-8.430407\du,-3.748760\du)--(-8.430407\du,-5.221387\du)--cycle;
        \pgfsetlinewidth{0.045000\du}
        \pgfsetdash{}{0pt}
        \pgfsetdash{}{0pt}
        \pgfsetmiterjoin
        \definecolor{dialinecolor}{rgb}{0.000000, 0.000000, 0.000000}
        \pgfsetstrokecolor{dialinecolor}
        \draw (-12.996782\du,-5.221387\du)--(-12.996782\du,-3.748760\du)--(-8.430407\du,-3.748760\du)--(-8.430407\du,-5.221387\du)--cycle;
        % setfont left to latex
        \definecolor{dialinecolor}{rgb}{0.000000, 0.000000, 0.000000}
        \pgfsetstrokecolor{dialinecolor}
        \node at (-10.713594\du,-4.392756\du){\scriptsize \textbf{Base::fun2()}};
        \definecolor{dialinecolor}{rgb}{1.000000, 0.752941, 0.796078}
        \pgfsetfillcolor{dialinecolor}
        \fill (5.613849\du,-8.160374\du)--(5.613849\du,-6.687746\du)--(10.180224\du,-6.687746\du)--(10.180224\du,-8.160374\du)--cycle;
        \pgfsetlinewidth{0.045000\du}
        \pgfsetdash{}{0pt}
        \pgfsetdash{}{0pt}
        \pgfsetmiterjoin
        \definecolor{dialinecolor}{rgb}{0.000000, 0.000000, 0.000000}
        \pgfsetstrokecolor{dialinecolor}
        \draw (5.613849\du,-8.160374\du)--(5.613849\du,-6.687746\du)--(10.180224\du,-6.687746\du)--(10.180224\du,-8.160374\du)--cycle;
        % setfont left to latex
        \definecolor{dialinecolor}{rgb}{0.000000, 0.000000, 0.000000}
        \pgfsetstrokecolor{dialinecolor}
        \node at (7.897036\du,-7.331743\du){\scriptsize \textbf{D1::\~{}D1()}};
        \definecolor{dialinecolor}{rgb}{1.000000, 0.752941, 0.796078}
        \pgfsetfillcolor{dialinecolor}
        \fill (5.613849\du,-6.685674\du)--(5.613849\du,-5.213047\du)--(10.180224\du,-5.213047\du)--(10.180224\du,-6.685674\du)--cycle;
        \pgfsetlinewidth{0.045000\du}
        \pgfsetdash{}{0pt}
        \pgfsetdash{}{0pt}
        \pgfsetmiterjoin
        \definecolor{dialinecolor}{rgb}{0.000000, 0.000000, 0.000000}
        \pgfsetstrokecolor{dialinecolor}
        \draw (5.613849\du,-6.685674\du)--(5.613849\du,-5.213047\du)--(10.180224\du,-5.213047\du)--(10.180224\du,-6.685674\du)--cycle;
        % setfont left to latex
        \definecolor{dialinecolor}{rgb}{0.000000, 0.000000, 0.000000}
        \pgfsetstrokecolor{dialinecolor}
        \node at (7.897036\du,-5.857043\du){\scriptsize \textbf{D1::fun1()}};
        \definecolor{dialinecolor}{rgb}{1.000000, 0.752941, 0.796078}
        \pgfsetfillcolor{dialinecolor}
        \fill (5.613849\du,-5.208565\du)--(5.613849\du,-3.735937\du)--(10.180224\du,-3.735937\du)--(10.180224\du,-5.208565\du)--cycle;
        \pgfsetlinewidth{0.045000\du}
        \pgfsetdash{}{0pt}
        \pgfsetdash{}{0pt}
        \pgfsetmiterjoin
        \definecolor{dialinecolor}{rgb}{0.000000, 0.000000, 0.000000}
        \pgfsetstrokecolor{dialinecolor}
        \draw (5.613849\du,-5.208565\du)--(5.613849\du,-3.735937\du)--(10.180224\du,-3.735937\du)--(10.180224\du,-5.208565\du)--cycle;
        % setfont left to latex
        \definecolor{dialinecolor}{rgb}{0.000000, 0.000000, 0.000000}
        \pgfsetstrokecolor{dialinecolor}
        \node at (7.897036\du,-4.379934\du){\scriptsize \textbf{D1::fun2()}};
        \definecolor{dialinecolor}{rgb}{0.564706, 0.933333, 0.564706}
        \pgfsetfillcolor{dialinecolor}
        \fill (-6.659825\du,-12.074415\du)--(-6.659825\du,-10.409415\du)--(-2.794100\du,-10.409415\du)--(-2.794100\du,-12.074415\du)--cycle;
        \pgfsetlinewidth{0.045000\du}
        \pgfsetdash{}{0pt}
        \pgfsetdash{}{0pt}
        \pgfsetmiterjoin
        \definecolor{dialinecolor}{rgb}{0.000000, 0.000000, 0.000000}
        \pgfsetstrokecolor{dialinecolor}
        \draw (-6.659825\du,-12.074415\du)--(-6.659825\du,-10.409415\du)--(-2.794100\du,-10.409415\du)--(-2.794100\du,-12.074415\du)--cycle;
        % setfont left to latex
        \definecolor{dialinecolor}{rgb}{0.000000, 0.000000, 0.000000}
        \pgfsetstrokecolor{dialinecolor}
        \node at (-4.726962\du,-11.118574\du){\scriptsize \textbf{m\_strData1}};
      \end{tikzpicture}

    \end{singlespace}
  \end{center}
  \caption{Base和D1的内存布局}
  \label{fig1}
\end{figure}

我们出问题的地方就在通过基类指针释放派生类的对象时,由于派生类的析构函数没有被调用,导致数据成员m\_strData2指向的内存没有被释放。

\section{问题总结}
虽然我们知道动态内存的申请和释放一定要成对,但对该问题来说有一个地方比较隐晦,就是在派生类中并没有显示的动态申请内存,而是成员变量string在动态申请,这也导致了在代码检视或问题排查的时候容易被忽略,最终将问题遗留下来。

\section{virtual背后的原理}
关于C++的虚函数,之前也写过一篇文章分析过,参见\href{http://blog.csdn.net/voiceofamerica/article/details/12692287}{这里}。这里再谈一下,为什么基类的析构函数定义为虚函数,通过基类指针就可以释放派生类的对象且派生类的析构函数会被正确的调用,而非虚函数则达不到这一效果。

首先来看Base的析构函数为非虚函数的情况下,函数mem\_leak\_test()反汇编后的结果,如代码\ref{lst3}所示:
\lstinputlisting[label=lst3,caption=非虚析构,language={[x86masm]Assembler}]{nonvirtual.s}

代码\ref{lst3}中的第8行调用new来申请内存,在11行调用类D1的构造函数对新申请的对象初始化,第17行调用Base的析构函数,第19行调用delete释放内存。可见,在整个函数的执行过程中,并没有调用D1的析构函数。

将Base的析构函数定义为虚函数后,函数mem\_leak\_test()反汇编后的结果如代码\ref{lst4}所示:
\lstinputlisting[label=lst4,caption=虚析构,language={[x86masm]Assembler}]{virtual.s}

反汇编后代码中的第8行调用new申请内存,第11行调用类D1的构造函数对新申请的对象初始化,第25行调用delete释放内存,这和非虚析构函数的情况是一样的。不同之处在于第21行,这里不再是调用Base的析构函数,而是call *\%eax。call *\%eax表示什么意思呢,见代码\ref{lst5}。
% \lstinputlisting[escapeinside=``,lineskip=-0.0em,label=lst5,caption=代码注释,language={[x86masm]Assembler}]{1.s}
\lstinputlisting[label=lst5,caption=代码注释,language={[x86masm]Assembler}]{1.s}

从代码\ref{lst5}的注释中可以看出,call *\%eax实际上是调用了类D1虚函数表中的第二个函数。D1虚函数表中的第二个函数即D1::\~{}D1(),如图\ref{fig1}所示\footnote{关于虚函数表中为什么有两个虚析构函数,参见\href{http://stackoverflow.com/questions/17960917/why-there-are-two-virtual-destructor-in-the-virtual-table-and-where-is-address-o}{这里}。},这样就达到了通过基类指针正确的释放派生类对象的目的。

\end{document}
%%%%%%%%%%%%%%%%%%%%%%%%% 正文部分结束%%%%%%%%%%%%%%%%%%%%%%

